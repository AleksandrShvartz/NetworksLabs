\documentclass[12pt,a4paper]{article}
\usepackage[utf8]{inputenc}
\usepackage[T2A]{fontenc}
\usepackage[russian]{babel}
\usepackage{amsmath,amssymb}
\usepackage{graphicx}
\usepackage{booktabs}
\usepackage{multirow}
\usepackage{geometry}
\usepackage{float}
\usepackage{caption}
\usepackage{hyperref}
\usepackage{pgfplots}
\pgfplotsset{compat=1.18}
\usepackage{listings}
\usepackage{xcolor}
\usepackage{tikz}

\geometry{left=2cm,right=2cm,top=2cm,bottom=2cm}

\lstset{
    language=C++,
    basicstyle=\ttfamily\small,
    keywordstyle=\color{blue},
    commentstyle=\color{green!50!black},
    stringstyle=\color{red},
    numbers=left,
    numberstyle=\tiny,
    breaklines=true,
    frame=single
}

\begin{document}

\begin{titlepage}
    \begin{center}
        \textsc{
            Санкт-Петербургский политехнический университет имени Петра Великого \\[5mm]
            Институт прикладной математики и механики\\[2mm]
            Высшая школа прикладной математики и физики
        }
        \vfill
        \textbf{\large
            Отчет \\
            по лабораторной работе \#2 \\
            по дисциплине \\
            ``Компьютерные сети''\\[5mm]
        }
        \textbf{\Large
            Реализация протокола динамической маршрутизации\\
            Open Shortest Path First\\
        }
    \end{center}

    \vfill
    \hfill
    \begin{minipage}{0.5\textwidth}
        Выполнил: \vspace{1ex} \\
		Студент: Шварц Александр \\
		Группа: 5040102/40201\\
    \end{minipage}

	\hfill
	\begin{minipage}{0.5\textwidth}
		Принял: \vspace{1ex} \\
		к. ф.-м. н., доцент \\
		Баженов Александр Николаевич
	\end{minipage}

    \vfill
    \begin{center}
        Санкт-Петербург \\
        \the\year{} г.
    \end{center}
\end{titlepage}

\tableofcontents
\newpage

% =================================================================
\section{Введение}

В данной работе рассматривается задача динамической маршрутизации в компьютерных сетях. За основу взят протокол OSPF (Open Shortest Path First), который относится к классу link-state протоколов: каждый маршрутизатор собирает информацию о всей топологии сети и на её основе самостоятельно строит таблицу маршрутизации, используя алгоритм Дейкстры.

Задача состоит в том, чтобы написать симулятор, в котором несколько маршрутизаторов обмениваются служебными сообщениями, узнают структуру сети и умеют пересылать данные по кратчайшему пути. Помимо этого, нужно посмотреть, как протокол справляется с отказами каналов --- как при одиночном разрыве, так и при случайных массовых сбоях.

Рассматриваются три топологии: линейная цепочка, кольцо и звезда.

% =================================================================
\section{Теоретические основы}

\subsection{Как работает OSPF}

Работа протокола разбивается на три фазы. Сначала каждый маршрутизатор составляет так называемое LSA (Link State Advertisement) --- по сути, список своих соседей с указанием стоимости канала до каждого из них. Затем эти LSA лавинообразно рассылаются по сети: получив чужое LSA, маршрутизатор запоминает его и пересылает дальше всем своим соседям. Когда новых LSA больше не приходит, у каждого узла есть полная база данных о состоянии каналов (LSDB). На третьем шаге каждый маршрутизатор запускает алгоритм Дейкстры на этой базе и получает дерево кратчайших путей до всех остальных узлов.

\subsection{Топологии}

В работе рассмотрены три варианта соединения маршрутизаторов:

\begin{itemize}
    \item \textbf{Линейная} --- узлы стоят цепочкой: $0{-}1{-}2{-}\ldots{-}(N{-}1)$. Каналов $N - 1$, диаметр сети $N - 1$. Самая простая и самая уязвимая к разрывам.
    \item \textbf{Кольцо} --- то же, но последний узел соединён с первым: $0{-}1{-}\ldots{-}(N{-}1){-}0$. Каналов $N$, диаметр $\lfloor N/2 \rfloor$. Появляется второй путь между любыми узлами.
    \item \textbf{Звезда} --- один центральный маршрутизатор (0), к которому подключены все остальные. Каналов $N - 1$, диаметр 2. Быстрая сходимость, но если центр откажет --- всё рассыплется.
\end{itemize}

\begin{figure}[H]
\centering
\begin{tikzpicture}[node distance=1.2cm, every node/.style={circle, draw, minimum size=0.7cm, font=\small}]
% Linear
\node at (-5.5, 1.5) [draw=none, minimum size=0] {\textbf{Линейная}};
\node (l0) at (-7, 0) {0};
\node (l1) at (-5.5, 0) {1};
\node (l2) at (-4, 0) {2};
\node (l3) at (-2.5, 0) {3};
\node (l4) at (-1, 0) {4};
\draw (l0)--(l1)--(l2)--(l3)--(l4);

% Ring
\node at (2.5, 1.5) [draw=none, minimum size=0] {\textbf{Кольцо}};
\node (r0) at (2.5, 0.8) {0};
\node (r1) at (3.7, 0.2) {1};
\node (r2) at (3.3, -1) {2};
\node (r3) at (1.7, -1) {3};
\node (r4) at (1.3, 0.2) {4};
\draw (r0)--(r1)--(r2)--(r3)--(r4)--(r0);

% Star
\node at (7.5, 1.5) [draw=none, minimum size=0] {\textbf{Звезда}};
\node (s0) at (7.5, 0) {0};
\node (s1) at (6, 0.7) {1};
\node (s2) at (6.5, -0.8) {2};
\node (s3) at (8.5, -0.8) {3};
\node (s4) at (9, 0.7) {4};
\draw (s0)--(s1) (s0)--(s2) (s0)--(s3) (s0)--(s4);
\end{tikzpicture}
\caption{Топологии сети (пример для $N = 5$)}
\label{fig:topologies}
\end{figure}

% =================================================================
\section{Реализация}

Программа написана на C++. Основные классы:

\begin{itemize}
    \item \texttt{Router} --- хранит свою LSDB, умеет генерировать LSA, принимать чужие и запускать Дейкстру для построения таблицы маршрутизации;
    \item \texttt{Network} --- собирает маршрутизаторы и каналы воедино, запускает процесс сходимости (итеративный обмен LSA до стабилизации) и умеет ломать/чинить каналы;
    \item \texttt{LSA} --- структура с id источника, порядковым номером и списком соседей;
    \item \texttt{Link} --- канал между двумя узлами, у которого есть флаг \texttt{active}.
\end{itemize}

Сходимость устроена просто: на каждом такте маршрутизаторы пересылают соседям все новые LSA. Как только за целый такт ни одна LSDB не обновилась --- считаем, что сеть сошлась, и каждый узел пересчитывает свою таблицу. Чтобы убедиться в правильности, после каждого эксперимента таблицы маршрутизации проверяются: для каждой пары узлов BFS-ом находится эталонный кратчайший путь и сравнивается с тем, что выдал Дейкстра.

% =================================================================
\section{Результаты экспериментов}

\subsection{Начальная сходимость}

Первым делом посмотрим, как быстро протокол сходится на каждой топологии при 10 маршрутизаторах и без каких-либо сбоев.

\begin{table}[H]
\centering
\caption{Начальная сходимость ($N = 10$)}
\label{tab:convergence}
\begin{tabular}{lccc}
\toprule
Топология & Тактов до сходимости & Достижимых пар & Средняя длина пути \\
\midrule
Линейная & 10 & 90/90 & 3.67 \\
Кольцо   & 6  & 90/90 & 2.78 \\
Звезда   & 3  & 90/90 & 1.80 \\
\bottomrule
\end{tabular}
\end{table}

Результат ожидаемый: скорость сходимости напрямую зависит от диаметра сети. В линейной топологии самый дальний LSA должен пройти через 9 промежуточных узлов --- отсюда 10 тактов (9 на рассылку + 1 проверочный, когда уже ничего нового не приходит). В звезде любой LSA за один шаг попадает в центр, а оттуда за второй шаг --- куда угодно, поэтому 3 тактов хватает при любом $N$.

\subsection{Масштабируемость}

Теперь увеличим число маршрутизаторов и посмотрим, как ведёт себя время сходимости.

\begin{table}[H]
\centering
\caption{Время сходимости и средняя длина пути для разных $N$}
\label{tab:scalability}
\begin{tabular}{l|cc|cc|cc}
\toprule
& \multicolumn{2}{c|}{Линейная} & \multicolumn{2}{c|}{Кольцо} & \multicolumn{2}{c}{Звезда} \\
$N$ & Такты & Ср. путь & Такты & Ср. путь & Такты & Ср. путь \\
\midrule
5   & 5  & 2.00  & 3  & 1.50  & 3 & 1.60 \\
10  & 10 & 3.67  & 6  & 2.78  & 3 & 1.80 \\
20  & 20 & 7.00  & 11 & 5.26  & 3 & 1.90 \\
50  & 50 & 17.00 & 26 & 12.76 & 3 & 1.96 \\
\bottomrule
\end{tabular}
\end{table}

\begin{figure}[H]
\centering
\begin{tikzpicture}
\begin{axis}[
    width=12cm, height=7cm,
    xlabel={Число маршрутизаторов $N$},
    ylabel={Тактов до сходимости},
    legend pos=north west,
    grid=major,
    ymin=0
]
\addplot[blue, mark=square*, thick] coordinates {
    (5,5) (10,10) (20,20) (50,50)
};
\addlegendentry{Линейная}

\addplot[red, mark=triangle*, thick] coordinates {
    (5,3) (10,6) (20,11) (50,26)
};
\addlegendentry{Кольцо}

\addplot[green!60!black, mark=diamond*, thick] coordinates {
    (5,3) (10,3) (20,3) (50,3)
};
\addlegendentry{Звезда}

\end{axis}
\end{tikzpicture}
\caption{Время сходимости OSPF в зависимости от $N$}
\label{fig:convergence_scale}
\end{figure}

Хорошо видно: для линейной топологии время сходимости растёт строго как $N$, для кольца --- примерно как $N/2$ (что совпадает с диаметром), а звезда остаётся на 3 тактах даже при 50 узлах.

\subsection{Таблицы маршрутизации}

Для наглядности приведём таблицу маршрутизации узла 0 в каждой топологии ($N = 10$).

\begin{table}[H]
\centering
\caption{Таблица маршрутизации узла 0}
\label{tab:routing_tables}
\small
\begin{tabular}{c|cc|cc|cc}
\toprule
& \multicolumn{2}{c|}{Линейная} & \multicolumn{2}{c|}{Кольцо} & \multicolumn{2}{c}{Звезда} \\
Назначение & Сл. хоп & Цена & Сл. хоп & Цена & Сл. хоп & Цена \\
\midrule
1 & 1 & 1 & 1 & 1 & 1 & 1 \\
2 & 1 & 2 & 1 & 2 & 2 & 1 \\
3 & 1 & 3 & 1 & 3 & 3 & 1 \\
4 & 1 & 4 & 1 & 4 & 4 & 1 \\
5 & 1 & 5 & 1 & 5 & 5 & 1 \\
6 & 1 & 6 & 9 & 4 & 6 & 1 \\
7 & 1 & 7 & 9 & 3 & 7 & 1 \\
8 & 1 & 8 & 9 & 2 & 8 & 1 \\
9 & 1 & 9 & 9 & 1 & 9 & 1 \\
\bottomrule
\end{tabular}
\end{table}

В линейной топологии у узла 0 выбора нет --- всё идёт через хоп 1. В кольце интереснее: до узлов 1--5 короче идти <<по часовой>> (через 1), а до узлов 6--9 --- <<против часовой>> (через 9). В звезде узел 0 и есть центр, поэтому каждый пункт назначения доступен напрямую за 1 хоп.

\subsection{Примеры маршрутов}

\begin{table}[H]
\centering
\caption{Маршруты от узла 0 к узлам 5 и 9}
\label{tab:routes}
\begin{tabular}{lll}
\toprule
Топология & $0 \to 5$ & $0 \to 9$ \\
\midrule
Линейная & $0 \to 1 \to 2 \to 3 \to 4 \to 5$ & $0 \to 1 \to 2 \to \ldots \to 9$ \\
Кольцо   & $0 \to 1 \to 2 \to 3 \to 4 \to 5$ & $0 \to 9$ \\
Звезда   & $0 \to 5$                          & $0 \to 9$ \\
\bottomrule
\end{tabular}
\end{table}

Маршрут $0 \to 9$ в кольце проходит в один хоп --- это показывает, что Дейкстра действительно выбирает более короткое направление.

\subsection{Стохастические разрывы связей}

Самый интересный эксперимент: берём рабочую сеть и случайным образом рвём каналы --- каждый независимо с вероятностью $p$. После этого маршрутизаторы пересходятся, и мы смотрим, какая доля пар узлов осталась достижимой. Результаты усреднены по 20 запускам.

\begin{table}[H]
\centering
\caption{Стохастические разрывы ($N = 10$)}
\label{tab:failures}
\small
\begin{tabular}{l|ccc|ccc|ccc}
\toprule
& \multicolumn{3}{c|}{Линейная (9 кан.)} & \multicolumn{3}{c|}{Кольцо (10 кан.)} & \multicolumn{3}{c}{Звезда (9 кан.)} \\
$p$ & Разр. & Связн. & Такты & Разр. & Связн. & Такты & Разр. & Связн. & Такты \\
\midrule
0.00 & 0.0 & 1.00 & 10.0 & 0.0 & 1.00 & 6.0 & 0.0 & 1.00 & 3.0 \\
0.10 & 0.9 & 0.73 & 8.3  & 0.9 & 0.92 & 8.2 & 0.9 & 0.84 & 3.0 \\
0.20 & 1.6 & 0.59 & 7.2  & 1.7 & 0.72 & 6.9 & 1.6 & 0.71 & 3.0 \\
0.30 & 2.5 & 0.43 & 5.8  & 2.8 & 0.52 & 5.8 & 2.5 & 0.58 & 3.0 \\
0.50 & 4.3 & 0.23 & 3.9  & 4.8 & 0.27 & 4.1 & 4.3 & 0.35 & 3.0 \\
\bottomrule
\end{tabular}
\end{table}

{\small \textbf{Разр.} --- среднее число разорванных каналов; \textbf{Связн.} --- доля достижимых пар; \textbf{Такты} --- время пересходимости.}

Важный момент: во всех 300 запусках (3 топологии $\times$ 5 вероятностей $\times$ 20 повторов) таблицы после пересходимости оказались корректными --- протокол ни разу не <<заблудился>>.

\begin{figure}[H]
\centering
\begin{tikzpicture}
\begin{axis}[
    width=12cm, height=7cm,
    xlabel={Вероятность разрыва $p$},
    ylabel={Доля достижимых пар},
    legend pos=south west,
    grid=major,
    ymin=0, ymax=1.05,
    xtick={0,0.1,0.2,0.3,0.5}
]
\addplot[blue, mark=square*, thick] coordinates {
    (0,1.00) (0.10,0.73) (0.20,0.59) (0.30,0.43) (0.50,0.23)
};
\addlegendentry{Линейная}

\addplot[red, mark=triangle*, thick] coordinates {
    (0,1.00) (0.10,0.92) (0.20,0.72) (0.30,0.52) (0.50,0.27)
};
\addlegendentry{Кольцо}

\addplot[green!60!black, mark=diamond*, thick] coordinates {
    (0,1.00) (0.10,0.84) (0.20,0.71) (0.30,0.58) (0.50,0.35)
};
\addlegendentry{Звезда}

\end{axis}
\end{tikzpicture}
\caption{Связность сети при стохастических разрывах}
\label{fig:reachability}
\end{figure}

При небольших $p$ лучше всех держится кольцо (92\% при $p=0{,}1$) --- логично, ведь там два пути между любыми узлами, и для потери связности нужно оборвать оба. Линейная топология теряет связность уже при одном разрыве, поэтому проседает сильнее всего (73\%).

Любопытно, что при $p=0{,}5$ звезда обгоняет кольцо (35\% против 27\%). Дело в том, что в звезде каждый периферийный узел зависит только от своего единственного канала до центра, а в кольце при половине оборванных каналов сеть рассыпается на множество мелких фрагментов.

\subsection{Разрыв и восстановление одного канала}

Отдельно проверим поведение при поломке и починке конкретного канала (0--1).

\begin{table}[H]
\centering
\caption{Разрыв и восстановление канала 0--1}
\label{tab:single_failure}
\begin{tabular}{llccc}
\toprule
Топология & Событие & Тактов & Достижимых пар & Корректность \\
\midrule
\multirow{3}{*}{Линейная} & Начальное состояние & 10 & 90 & Да \\
 & Разрыв & 9 & 72 & Да \\
 & Восстановление & 10 & 90 & Да \\
\midrule
\multirow{3}{*}{Кольцо} & Начальное состояние & 6 & 90 & Да \\
 & Разрыв & 10 & 90 & Да \\
 & Восстановление & 6 & 90 & Да \\
\midrule
\multirow{3}{*}{Звезда} & Начальное состояние & 3 & 90 & Да \\
 & Разрыв & 3 & 72 & Да \\
 & Восстановление & 3 & 90 & Да \\
\bottomrule
\end{tabular}
\end{table}

Здесь хорошо видна разница между топологиями. В кольце после разрыва канала 0--1 все 90 пар по-прежнему достижимы --- трафик просто пошёл в обход. Правда, сеть фактически стала линейной, и время пересходимости подскочило с 6 до 10 тактов. В линейной топологии и в звезде разрыв канала 0--1 отрезает узел 1, и 18 пар (с участием узла 1) становятся недостижимыми.

% =================================================================
\section{Анализ}

У каждой топологии свои сильные и слабые стороны.

\textbf{Звезда} --- чемпион по скорости: сходится за 3 такта при любом $N$, маршруты самые короткие. Но всё завязано на центральный узел. Если он упадёт, сеть перестаёт существовать. Для небольших сетей с надёжным центром это хороший выбор.

\textbf{Кольцо} --- золотая середина. Единичный разрыв канала не страшен: всегда есть обходной путь. Время сходимости умеренное ($\approx N/2$). Из минусов --- при двух и более разрывах всё уже не так радужно.

\textbf{Линейная топология} --- самая хрупкая. Любой единственный разрыв рубит сеть на две части. На практике чистая линейная топология встречается разве что в цепочках повторителей.

Что касается устойчивости к массовым сбоям: при $p=0{,}1$ кольцо сохраняет 92\% связности, звезда 84\%, линейная --- только 73\%. При $p=0{,}3$ все топологии уже ниже 60\%, что для реальной сети означало бы серьёзные проблемы.

% =================================================================
\section{Выводы}

\begin{enumerate}
    \item Реализован симулятор протокола OSPF на C++ с рассылкой LSA, алгоритмом Дейкстры и маршрутизацией по хопам. Корректность проверена на всех экспериментах.

    \item Скорость сходимости определяется диаметром сети: $N$ тактов для линейной, $\approx N/2$ для кольца, 3 такта для звезды вне зависимости от размера.

    \item Кольцо --- единственная из рассмотренных топологий, которая переживает одиночный разрыв без потери связности.

    \item Звезда быстрее всех сходится и даёт кратчайшие маршруты, но полностью зависит от центрального узла.

    \item При случайных разрывах с $p \geq 0{,}3$ все три топологии теряют больше 40\% связности --- в реальных сетях это аргумент в пользу избыточных каналов и более сложных топологий (например, полносвязных или ячеистых).
\end{enumerate}

% =================================================================
\section*{Приложение: исходный код}
\addcontentsline{toc}{section}{Приложение: исходный код}

Исходный код симулятора доступен в репозитории:

\url{https://github.com/AleksandrShvartz/NetworksLabs}

\end{document}
