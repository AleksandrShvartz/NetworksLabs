\documentclass[12pt,a4paper]{article}
\usepackage[utf8]{inputenc}
\usepackage[T2A]{fontenc}
\usepackage[russian]{babel}
\usepackage{amsmath,amssymb}
\usepackage{graphicx}
\usepackage{booktabs}
\usepackage{multirow}
\usepackage{geometry}
\usepackage{float}
\usepackage{caption}
\usepackage{hyperref}
\usepackage{pgfplots}
\pgfplotsset{compat=1.18}
\usepackage{listings}
\usepackage{xcolor}
\usepackage{tikz}

\geometry{left=2cm,right=2cm,top=2cm,bottom=2cm}

\lstset{
    language=C++,
    basicstyle=\ttfamily\small,
    keywordstyle=\color{blue},
    commentstyle=\color{green!50!black},
    stringstyle=\color{red},
    numbers=left,
    numberstyle=\tiny,
    breaklines=true,
    frame=single
}

\begin{document}

\begin{titlepage}
    \begin{center}
        \textsc{
            Санкт-Петербургский политехнический университет имени Петра Великого \\[5mm]
            Институт прикладной математики и механики\\[2mm]
            Высшая школа прикладной математики и физики
        }
        \vfill
        \textbf{\large
            Отчет \\
            по лабораторной работе \#2 \\
            по дисциплине \\
            ``Компьютерные сети''\\[5mm]
        }
        \textbf{\Large
            Реализация протокола динамической маршрутизации\\
            Open Shortest Path First\\
        }
    \end{center}

    \vfill
    \hfill
    \begin{minipage}{0.5\textwidth}
        Выполнил: \vspace{1ex} \\
		Студент: Шварц Александр \\
		Группа: 5040102/40201\\
    \end{minipage}

	\hfill
	\begin{minipage}{0.5\textwidth}
		Принял: \vspace{1ex} \\
		к. ф.-м. н., доцент \\
		Баженов Александр Николаевич
	\end{minipage}

    \vfill
    \begin{center}
        Санкт-Петербург \\
        \the\year{} г.
    \end{center}
\end{titlepage}

\tableofcontents
\newpage

% =================================================================
\section{Введение}

Протокол OSPF (Open Shortest Path First) --- протокол динамической маршрутизации, основанный на алгоритме состояния канала (link-state). Каждый маршрутизатор поддерживает полную карту сети (Link State Database, LSDB) и вычисляет кратчайшие пути с помощью алгоритма Дейкстры (SPF --- Shortest Path First).

Цель работы --- реализовать систему взаимодействующих маршрутизаторов с протоколом OSPF, исследовать сходимость протокола на различных топологиях (линейная, кольцо, звезда) и проанализировать перестройку таблиц маршрутизации при стохастических разрывах связей.

% =================================================================
\section{Теоретические основы}

\subsection{Протокол OSPF}

OSPF работает в три этапа:
\begin{enumerate}
    \item \textbf{Генерация LSA.} Каждый маршрутизатор формирует объявление состояния канала (Link State Advertisement, LSA), содержащее список его активных соседей с указанием стоимости каналов.
    \item \textbf{Рассылка LSA (Flooding).} LSA рассылаются всем маршрутизаторам сети через механизм лавинной рассылки: каждый маршрутизатор пересылает новые LSA всем своим соседям.
    \item \textbf{Вычисление SPF.} После формирования полной LSDB каждый маршрутизатор независимо запускает алгоритм Дейкстры для построения дерева кратчайших путей и заполнения таблицы маршрутизации.
\end{enumerate}

\subsection{Рассматриваемые топологии}

\begin{itemize}
    \item \textbf{Линейная:} маршрутизаторы соединены последовательно ($0{-}1{-}2{-}\ldots{-}(N{-}1)$). Количество каналов: $N - 1$. Диаметр сети: $N - 1$.
    \item \textbf{Кольцо:} маршрутизаторы образуют цикл ($0{-}1{-}\ldots{-}(N{-}1){-}0$). Количество каналов: $N$. Диаметр сети: $\lfloor N/2 \rfloor$.
    \item \textbf{Звезда:} центральный маршрутизатор (0) соединён со всеми остальными. Количество каналов: $N - 1$. Диаметр сети: $2$.
\end{itemize}

\begin{figure}[H]
\centering
\begin{tikzpicture}[node distance=1.2cm, every node/.style={circle, draw, minimum size=0.7cm, font=\small}]
% Linear
\node at (-5.5, 1.5) [draw=none, minimum size=0] {\textbf{Линейная}};
\node (l0) at (-7, 0) {0};
\node (l1) at (-5.5, 0) {1};
\node (l2) at (-4, 0) {2};
\node (l3) at (-2.5, 0) {3};
\node (l4) at (-1, 0) {4};
\draw (l0)--(l1)--(l2)--(l3)--(l4);

% Ring
\node at (2.5, 1.5) [draw=none, minimum size=0] {\textbf{Кольцо}};
\node (r0) at (2.5, 0.8) {0};
\node (r1) at (3.7, 0.2) {1};
\node (r2) at (3.3, -1) {2};
\node (r3) at (1.7, -1) {3};
\node (r4) at (1.3, 0.2) {4};
\draw (r0)--(r1)--(r2)--(r3)--(r4)--(r0);

% Star
\node at (7.5, 1.5) [draw=none, minimum size=0] {\textbf{Звезда}};
\node (s0) at (7.5, 0) {0};
\node (s1) at (6, 0.7) {1};
\node (s2) at (6.5, -0.8) {2};
\node (s3) at (8.5, -0.8) {3};
\node (s4) at (9, 0.7) {4};
\draw (s0)--(s1) (s0)--(s2) (s0)--(s3) (s0)--(s4);
\end{tikzpicture}
\caption{Рассматриваемые топологии сети (пример для $N = 5$)}
\label{fig:topologies}
\end{figure}

% =================================================================
\section{Реализация}

Система реализована на языке C++ и включает следующие компоненты:

\begin{itemize}
    \item \texttt{Router} --- маршрутизатор с собственной LSDB, таблицей маршрутизации и реализацией алгоритма Дейкстры;
    \item \texttt{Network} --- сеть, управляющая топологией, каналами и процессом сходимости;
    \item \texttt{LSA} --- структура объявления состояния канала (идентификатор источника, номер последовательности, список соседей);
    \item \texttt{Link} --- канал связи с возможностью деактивации.
\end{itemize}

\textbf{Модель симуляции.} Сходимость моделируется итеративно: на каждом такте маршрутизаторы обмениваются LSA с непосредственными соседями. Процесс продолжается до тех пор, пока ни одна LSDB не обновляется. После этого каждый маршрутизатор запускает алгоритм Дейкстры. Корректность таблиц маршрутизации верифицируется сравнением с эталонными кратчайшими путями, полученными BFS.

% =================================================================
\section{Результаты экспериментов}

\subsection{Начальная сходимость}

\begin{table}[H]
\centering
\caption{Начальная сходимость протокола ($N = 10$)}
\label{tab:convergence}
\begin{tabular}{lccc}
\toprule
Топология & Тактов до сходимости & Достижимых пар & Средняя длина пути \\
\midrule
Линейная & 10 & 90/90 & 3.67 \\
Кольцо   & 6  & 90/90 & 2.78 \\
Звезда   & 3  & 90/90 & 1.80 \\
\bottomrule
\end{tabular}
\end{table}

Время сходимости определяется диаметром сети: LSA от самого удалённого маршрутизатора должен достичь всех остальных узлов. В линейной топологии диаметр равен $N - 1 = 9$, поэтому сходимость занимает 10 тактов (9 тактов рассылки + 1 проверочный). В звезде диаметр равен 2, и сходимость наступает за 3 такта.

\subsection{Масштабируемость}

\begin{table}[H]
\centering
\caption{Время сходимости и средняя длина пути при различном числе маршрутизаторов}
\label{tab:scalability}
\begin{tabular}{l|cc|cc|cc}
\toprule
& \multicolumn{2}{c|}{Линейная} & \multicolumn{2}{c|}{Кольцо} & \multicolumn{2}{c}{Звезда} \\
$N$ & Такты & Ср. путь & Такты & Ср. путь & Такты & Ср. путь \\
\midrule
5   & 5  & 2.00  & 3  & 1.50  & 3 & 1.60 \\
10  & 10 & 3.67  & 6  & 2.78  & 3 & 1.80 \\
20  & 20 & 7.00  & 11 & 5.26  & 3 & 1.90 \\
50  & 50 & 17.00 & 26 & 12.76 & 3 & 1.96 \\
\bottomrule
\end{tabular}
\end{table}

\begin{figure}[H]
\centering
\begin{tikzpicture}
\begin{axis}[
    width=12cm, height=7cm,
    xlabel={Число маршрутизаторов $N$},
    ylabel={Тактов до сходимости},
    legend pos=north west,
    grid=major,
    ymin=0
]
\addplot[blue, mark=square*, thick] coordinates {
    (5,5) (10,10) (20,20) (50,50)
};
\addlegendentry{Линейная}

\addplot[red, mark=triangle*, thick] coordinates {
    (5,3) (10,6) (20,11) (50,26)
};
\addlegendentry{Кольцо}

\addplot[green!60!black, mark=diamond*, thick] coordinates {
    (5,3) (10,3) (20,3) (50,3)
};
\addlegendentry{Звезда}

\end{axis}
\end{tikzpicture}
\caption{Время сходимости OSPF в зависимости от числа маршрутизаторов}
\label{fig:convergence_scale}
\end{figure}

Время сходимости линейной топологии растёт как $O(N)$, кольца --- как $O(N/2)$, а звезды остаётся \emph{постоянным} ($3$ такта) независимо от $N$, поскольку любой LSA достигает всех узлов за 2 шага через центральный маршрутизатор.

\subsection{Таблицы маршрутизации}

Пример таблиц маршрутизации для маршрутизатора 0 ($N = 10$):

\begin{table}[H]
\centering
\caption{Таблица маршрутизации маршрутизатора 0 для разных топологий}
\label{tab:routing_tables}
\small
\begin{tabular}{c|cc|cc|cc}
\toprule
& \multicolumn{2}{c|}{Линейная} & \multicolumn{2}{c|}{Кольцо} & \multicolumn{2}{c}{Звезда} \\
Назначение & Сл. хоп & Цена & Сл. хоп & Цена & Сл. хоп & Цена \\
\midrule
1 & 1 & 1 & 1 & 1 & 1 & 1 \\
2 & 1 & 2 & 1 & 2 & 2 & 1 \\
3 & 1 & 3 & 1 & 3 & 3 & 1 \\
4 & 1 & 4 & 1 & 4 & 4 & 1 \\
5 & 1 & 5 & 1 & 5 & 5 & 1 \\
6 & 1 & 6 & 9 & 4 & 6 & 1 \\
7 & 1 & 7 & 9 & 3 & 7 & 1 \\
8 & 1 & 8 & 9 & 2 & 8 & 1 \\
9 & 1 & 9 & 9 & 1 & 9 & 1 \\
\bottomrule
\end{tabular}
\end{table}

В линейной топологии все пакеты от маршрутизатора 0 направляются через следующий хоп 1. В кольце маршрутизатор 0 выбирает кратчайшее направление: через хоп 1 для узлов 1--5, через хоп 9 для узлов 6--9. В звезде все пункты назначения достижимы напрямую через центр (хоп = назначение), с ценой~1.

\subsection{Примеры маршрутов}

\begin{table}[H]
\centering
\caption{Маршруты от узла 0 к узлам 5 и 9}
\label{tab:routes}
\begin{tabular}{lll}
\toprule
Топология & $0 \to 5$ & $0 \to 9$ \\
\midrule
Линейная & $0 \to 1 \to 2 \to 3 \to 4 \to 5$ & $0 \to 1 \to 2 \to \ldots \to 9$ \\
Кольцо   & $0 \to 1 \to 2 \to 3 \to 4 \to 5$ & $0 \to 9$ \\
Звезда   & $0 \to 5$                          & $0 \to 9$ \\
\bottomrule
\end{tabular}
\end{table}

\subsection{Стохастические разрывы связей}

Каждый активный канал разрывается независимо с вероятностью $p$, после чего маршрутизаторы перестраивают таблицы. Результаты усреднены по 20 запускам.

\begin{table}[H]
\centering
\caption{Влияние стохастических разрывов на связность сети ($N = 10$)}
\label{tab:failures}
\small
\begin{tabular}{l|ccc|ccc|ccc}
\toprule
& \multicolumn{3}{c|}{Линейная (9 каналов)} & \multicolumn{3}{c|}{Кольцо (10 каналов)} & \multicolumn{3}{c}{Звезда (9 каналов)} \\
$p$ & Разр. & Связн. & Такты & Разр. & Связн. & Такты & Разр. & Связн. & Такты \\
\midrule
0.00 & 0.0 & 1.00 & 10.0 & 0.0 & 1.00 & 6.0 & 0.0 & 1.00 & 3.0 \\
0.10 & 0.9 & 0.73 & 8.3  & 0.9 & 0.92 & 8.2 & 0.9 & 0.84 & 3.0 \\
0.20 & 1.6 & 0.59 & 7.2  & 1.7 & 0.72 & 6.9 & 1.6 & 0.71 & 3.0 \\
0.30 & 2.5 & 0.43 & 5.8  & 2.8 & 0.52 & 5.8 & 2.5 & 0.58 & 3.0 \\
0.50 & 4.3 & 0.23 & 3.9  & 4.8 & 0.27 & 4.1 & 4.3 & 0.35 & 3.0 \\
\bottomrule
\end{tabular}
\end{table}

{\small \textbf{Разр.} --- среднее число разорванных каналов; \textbf{Связн.} --- доля достижимых пар; \textbf{Такты} --- время пересходимости.}

Во всех экспериментах таблицы маршрутизации после пересходимости были \emph{корректны} (проверено сравнением с эталонными кратчайшими путями).

\begin{figure}[H]
\centering
\begin{tikzpicture}
\begin{axis}[
    width=12cm, height=7cm,
    xlabel={Вероятность разрыва $p$},
    ylabel={Доля достижимых пар},
    legend pos=south west,
    grid=major,
    ymin=0, ymax=1.05,
    xtick={0,0.1,0.2,0.3,0.5}
]
\addplot[blue, mark=square*, thick] coordinates {
    (0,1.00) (0.10,0.73) (0.20,0.59) (0.30,0.43) (0.50,0.23)
};
\addlegendentry{Линейная}

\addplot[red, mark=triangle*, thick] coordinates {
    (0,1.00) (0.10,0.92) (0.20,0.72) (0.30,0.52) (0.50,0.27)
};
\addlegendentry{Кольцо}

\addplot[green!60!black, mark=diamond*, thick] coordinates {
    (0,1.00) (0.10,0.84) (0.20,0.71) (0.30,0.58) (0.50,0.35)
};
\addlegendentry{Звезда}

\end{axis}
\end{tikzpicture}
\caption{Доля достижимых пар при стохастических разрывах связей}
\label{fig:reachability}
\end{figure}

\subsection{Разрыв и восстановление одиночного канала}

\begin{table}[H]
\centering
\caption{Поведение протокола при разрыве и восстановлении одного канала}
\label{tab:single_failure}
\begin{tabular}{llccc}
\toprule
Топология & Событие & Тактов & Достижимых пар & Таблицы корректны \\
\midrule
\multirow{3}{*}{Линейная} & Начальное & 10 & 90 & Да \\
 & Разрыв канала 0--1 & 9 & 72 & Да \\
 & Восстановление & 10 & 90 & Да \\
\midrule
\multirow{3}{*}{Кольцо} & Начальное & 6 & 90 & Да \\
 & Разрыв канала 0--1 & 10 & 90 & Да \\
 & Восстановление & 6 & 90 & Да \\
\midrule
\multirow{3}{*}{Звезда} & Начальное & 3 & 90 & Да \\
 & Разрыв канала 0--1 & 3 & 72 & Да \\
 & Восстановление & 3 & 90 & Да \\
\bottomrule
\end{tabular}
\end{table}

Ключевое наблюдение: при разрыве канала в кольцевой топологии сеть \emph{не теряет связность} (все 90 пар остаются достижимыми), так как кольцо обеспечивает альтернативный путь. Однако время пересходимости возрастает с 6 до 10 тактов, поскольку сеть фактически превращается в линейную. В линейной и звёздной топологиях разрыв канала изолирует узел 1 (72 из 90 пар).

% =================================================================
\section{Анализ результатов}

\subsection{Сравнение топологий}

\textbf{Звезда} обеспечивает минимальное время сходимости ($O(1)$) и кратчайшие маршруты (средняя длина $\approx 2$), но имеет единую точку отказа --- центральный маршрутизатор. При его выходе из строя сеть полностью распадается.

\textbf{Кольцо} обеспечивает отказоустойчивость при одиночных разрывах: любой единичный разрыв канала не нарушает связность. Время сходимости составляет $O(N/2)$.

\textbf{Линейная} топология наиболее уязвима: любой разрыв канала разделяет сеть на два несвязных сегмента. Время сходимости максимально --- $O(N)$.

\subsection{Устойчивость к разрывам}

При вероятности разрыва $p = 0{,}10$ кольцо сохраняет $92\%$ достижимых пар (против $73\%$ у линейной топологии), благодаря наличию двух путей между любыми узлами. Звезда занимает промежуточное положение ($84\%$).

При высоких вероятностях разрыва ($p = 0{,}50$) все топологии значительно теряют связность. Звезда оказывается более устойчивой ($35\%$) по сравнению с линейной ($23\%$) и кольцом ($27\%$), поскольку каждый узел в звезде напрямую связан с центром и независим от остальных каналов.

\subsection{Корректность пересходимости}

Во всех проведённых экспериментах (все топологии, все вероятности разрывов, 20 запусков на точку) таблицы маршрутизации после пересходимости были корректны. Это подтверждает правильность реализации механизмов лавинной рассылки LSA и алгоритма SPF.

% =================================================================
\section{Выводы}

\begin{enumerate}
    \item Реализован протокол OSPF с рассылкой LSA, алгоритмом Дейкстры и hop-by-hop маршрутизацией.

    \item \textbf{Время сходимости} определяется диаметром сети: $O(N)$ для линейной, $O(N/2)$ для кольца, $O(1)$ для звезды.

    \item \textbf{Кольцо обеспечивает наилучшую отказоустойчивость} при одиночных разрывах: связность сохраняется полностью за счёт альтернативного пути.

    \item \textbf{Звезда наиболее эффективна} по времени сходимости и длине маршрутов, но уязвима к отказу центрального узла.

    \item Протокол корректно перестраивает таблицы маршрутизации при стохастических разрывах любой конфигурации.

    \item При высоких вероятностях разрыва ($p \geq 0{,}30$) связность всех топологий падает ниже $60\%$, что указывает на необходимость резервных каналов в реальных сетях.
\end{enumerate}

% =================================================================
\section*{Приложение: исходный код}
\addcontentsline{toc}{section}{Приложение: исходный код}

Исходный код симулятора доступен в репозитории:

\url{https://github.com/AleksandrShvartz/NetworksLabs}

\end{document}
