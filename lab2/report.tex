\documentclass[12pt,a4paper]{article}
\usepackage[utf8]{inputenc}
\usepackage[T2A]{fontenc}
\usepackage[russian]{babel}
\usepackage{amsmath,amssymb}
\usepackage{graphicx}
\usepackage{booktabs}
\usepackage{multirow}
\usepackage{geometry}
\usepackage{float}
\usepackage{caption}
\usepackage{hyperref}
\usepackage{pgfplots}
\pgfplotsset{compat=1.18}
\usepackage{listings}
\usepackage{xcolor}
\usepackage{tikz}

\geometry{left=2cm,right=2cm,top=2cm,bottom=2cm}

\lstset{
    language=C++,
    basicstyle=\ttfamily\small,
    keywordstyle=\color{blue},
    commentstyle=\color{green!50!black},
    stringstyle=\color{red},
    numbers=left,
    numberstyle=\tiny,
    breaklines=true,
    frame=single
}

\begin{document}

\begin{titlepage}
    \begin{center}
        \textsc{
            Санкт-Петербургский политехнический университет имени Петра Великого \\[5mm]
            Институт прикладной математики и механики\\[2mm]
            Высшая школа прикладной математики и физики
        }
        \vfill
        \textbf{\large
            Отчет \\
            по лабораторной работе \#2 \\
            по дисциплине \\
            ``Компьютерные сети''\\[5mm]
        }
        \textbf{\Large
            Реализация протокола динамической маршрутизации\\
            Open Shortest Path First\\
        }
    \end{center}

    \vfill
    \hfill
    \begin{minipage}{0.5\textwidth}
        Выполнил: \vspace{1ex} \\
		Студент: Шварц Александр \\
		Группа: 5040102/40201\\
    \end{minipage}

	\hfill
	\begin{minipage}{0.5\textwidth}
		Принял: \vspace{1ex} \\
		к. ф.-м. н., доцент \\
		Баженов Александр Николаевич
	\end{minipage}

    \vfill
    \begin{center}
        Санкт-Петербург \\
        \the\year{} г.
    \end{center}
\end{titlepage}

\tableofcontents
\newpage

% =================================================================
\section{Введение}

Протокол OSPF (Open Shortest Path First) относится к классу link-state протоколов маршрутизации: каждый маршрутизатор собирает информацию о топологии сети и самостоятельно вычисляет таблицу маршрутизации с помощью алгоритма Дейкстры.

Цель работы --- реализовать симулятор протокола OSPF, в котором маршрутизаторы обмениваются служебными сообщениями (LSA), строят базу данных о состоянии каналов и выполняют маршрутизацию по кратчайшему пути. Кроме того, исследуется поведение протокола при отказах каналов: одиночных и массовых стохастических.

Эксперименты проводятся на трёх топологиях: линейной, кольцевой и звёздной.

% =================================================================
\section{Теоретические основы}

\subsection{Протокол OSPF}

Работа протокола включает три этапа. На первом этапе каждый маршрутизатор формирует LSA (Link State Advertisement) --- сообщение, содержащее идентификатор маршрутизатора и список его соседей с указанием стоимости каналов. На втором этапе LSA лавинообразно распространяются по сети: при получении нового LSA маршрутизатор сохраняет его в локальную базу данных (LSDB) и ретранслирует всем соседям. После завершения обмена каждый узел располагает полной копией LSDB. На третьем этапе маршрутизатор применяет алгоритм Дейкстры к LSDB и строит дерево кратчайших путей до всех остальных узлов сети.

\subsection{Топологии}

В работе рассмотрены три топологии:

\begin{itemize}
    \item \textbf{Линейная} --- узлы соединены последовательно: $0{-}1{-}2{-}\ldots{-}(N{-}1)$. Число каналов $N - 1$, диаметр сети $N - 1$.
    \item \textbf{Кольцо} --- аналогична линейной, но с дополнительным каналом между узлами $0$ и $N{-}1$. Число каналов $N$, диаметр $\lfloor N/2 \rfloor$. Между любой парой узлов существуют два независимых пути.
    \item \textbf{Звезда} --- выделенный центральный маршрутизатор (узел 0) соединён со всеми остальными. Число каналов $N - 1$, диаметр 2.
\end{itemize}

\begin{figure}[H]
\centering
\begin{tikzpicture}[node distance=1.2cm, every node/.style={circle, draw, minimum size=0.7cm, font=\small}]
% Linear
\node at (-5.5, 1.5) [draw=none, minimum size=0] {\textbf{Линейная}};
\node (l0) at (-7, 0) {0};
\node (l1) at (-5.5, 0) {1};
\node (l2) at (-4, 0) {2};
\node (l3) at (-2.5, 0) {3};
\node (l4) at (-1, 0) {4};
\draw (l0)--(l1)--(l2)--(l3)--(l4);

% Ring
\node at (2.5, 1.5) [draw=none, minimum size=0] {\textbf{Кольцо}};
\node (r0) at (2.5, 0.8) {0};
\node (r1) at (3.7, 0.2) {1};
\node (r2) at (3.3, -1) {2};
\node (r3) at (1.7, -1) {3};
\node (r4) at (1.3, 0.2) {4};
\draw (r0)--(r1)--(r2)--(r3)--(r4)--(r0);

% Star
\node at (7.5, 1.5) [draw=none, minimum size=0] {\textbf{Звезда}};
\node (s0) at (7.5, 0) {0};
\node (s1) at (6, 0.7) {1};
\node (s2) at (6.5, -0.8) {2};
\node (s3) at (8.5, -0.8) {3};
\node (s4) at (9, 0.7) {4};
\draw (s0)--(s1) (s0)--(s2) (s0)--(s3) (s0)--(s4);
\end{tikzpicture}
\caption{Топологии сети (пример для $N = 5$)}
\label{fig:topologies}
\end{figure}

% =================================================================
\section{Реализация}

Симулятор реализован на языке C++ и включает следующие компоненты:

\begin{itemize}
    \item \texttt{Router} --- маршрутизатор с локальной LSDB; генерирует собственные LSA, обрабатывает входящие и вычисляет таблицу маршрутизации алгоритмом Дейкстры;
    \item \texttt{Network} --- управляет множеством маршрутизаторов и каналов связи, реализует цикл сходимости и операции разрыва/восстановления каналов;
    \item \texttt{LSA} --- структура, содержащая идентификатор источника, порядковый номер и список соседей;
    \item \texttt{Link} --- канал между двумя узлами с флагом активности.
\end{itemize}

Моделирование выполняется пошагово: на каждом такте маршрутизаторы передают соседям все новые LSA. Сходимость фиксируется, когда за полный такт ни одна LSDB не обновилась; после этого каждый узел пересчитывает таблицу маршрутизации. Для верификации результатов после каждого эксперимента выполняется проверка: для всех пар узлов BFS-ом определяется эталонный кратчайший путь и сравнивается с результатом алгоритма Дейкстры.

% =================================================================
\section{Результаты экспериментов}

\subsection{Начальная сходимость}

В таблице~\ref{tab:convergence} приведены результаты начальной сходимости для $N = 10$ без отказов каналов.

\begin{table}[H]
\centering
\caption{Начальная сходимость ($N = 10$)}
\label{tab:convergence}
\begin{tabular}{lccc}
\toprule
Топология & Тактов до сходимости & Достижимых пар & Средняя длина пути \\
\midrule
Линейная & 10 & 90/90 & 3.67 \\
Кольцо   & 6  & 90/90 & 2.78 \\
Звезда   & 3  & 90/90 & 1.80 \\
\bottomrule
\end{tabular}
\end{table}

Скорость сходимости определяется диаметром сети. В линейной топологии LSA от крайнего узла должно пройти через $N - 1$ промежуточных маршрутизаторов, что даёт $N$ тактов (включая завершающий такт без обновлений). В звезде любое LSA достигает центрального узла за один такт, а от него --- всех периферийных за второй, поэтому сходимость составляет 3 такта при любом $N$.

\subsection{Масштабируемость}

Зависимость времени сходимости от числа маршрутизаторов представлена в таблице~\ref{tab:scalability} и на рис.~\ref{fig:convergence_scale}.

\begin{table}[H]
\centering
\caption{Время сходимости и средняя длина пути для разных $N$}
\label{tab:scalability}
\begin{tabular}{l|cc|cc|cc}
\toprule
& \multicolumn{2}{c|}{Линейная} & \multicolumn{2}{c|}{Кольцо} & \multicolumn{2}{c}{Звезда} \\
$N$ & Такты & Ср. путь & Такты & Ср. путь & Такты & Ср. путь \\
\midrule
5   & 5  & 2.00  & 3  & 1.50  & 3 & 1.60 \\
10  & 10 & 3.67  & 6  & 2.78  & 3 & 1.80 \\
20  & 20 & 7.00  & 11 & 5.26  & 3 & 1.90 \\
50  & 50 & 17.00 & 26 & 12.76 & 3 & 1.96 \\
\bottomrule
\end{tabular}
\end{table}

\begin{figure}[H]
\centering
\begin{tikzpicture}
\begin{axis}[
    width=12cm, height=7cm,
    xlabel={Число маршрутизаторов $N$},
    ylabel={Тактов до сходимости},
    legend pos=north west,
    grid=major,
    ymin=0
]
\addplot[blue, mark=square*, thick] coordinates {
    (5,5) (10,10) (20,20) (50,50)
};
\addlegendentry{Линейная}

\addplot[red, mark=triangle*, thick] coordinates {
    (5,3) (10,6) (20,11) (50,26)
};
\addlegendentry{Кольцо}

\addplot[green!60!black, mark=diamond*, thick] coordinates {
    (5,3) (10,3) (20,3) (50,3)
};
\addlegendentry{Звезда}

\end{axis}
\end{tikzpicture}
\caption{Время сходимости OSPF в зависимости от $N$}
\label{fig:convergence_scale}
\end{figure}

Для линейной топологии время сходимости растёт как $O(N)$, для кольца --- как $O(N/2)$, что соответствует диаметру. Звезда сохраняет постоянное время сходимости (3 такта) при любом числе узлов.

\subsection{Таблицы маршрутизации}

В таблице~\ref{tab:routing_tables} приведены таблицы маршрутизации узла~0 для каждой топологии при $N = 10$.

\begin{table}[H]
\centering
\caption{Таблица маршрутизации узла 0}
\label{tab:routing_tables}
\small
\begin{tabular}{c|cc|cc|cc}
\toprule
& \multicolumn{2}{c|}{Линейная} & \multicolumn{2}{c|}{Кольцо} & \multicolumn{2}{c}{Звезда} \\
Назначение & Сл. хоп & Цена & Сл. хоп & Цена & Сл. хоп & Цена \\
\midrule
1 & 1 & 1 & 1 & 1 & 1 & 1 \\
2 & 1 & 2 & 1 & 2 & 2 & 1 \\
3 & 1 & 3 & 1 & 3 & 3 & 1 \\
4 & 1 & 4 & 1 & 4 & 4 & 1 \\
5 & 1 & 5 & 1 & 5 & 5 & 1 \\
6 & 1 & 6 & 9 & 4 & 6 & 1 \\
7 & 1 & 7 & 9 & 3 & 7 & 1 \\
8 & 1 & 8 & 9 & 2 & 8 & 1 \\
9 & 1 & 9 & 9 & 1 & 9 & 1 \\
\bottomrule
\end{tabular}
\end{table}

В линейной топологии единственным следующим хопом для узла~0 является узел~1. В кольце маршруты разделяются: к узлам 1--5 трафик направляется через узел~1, к узлам 6--9 --- через узел~9, что соответствует выбору кратчайшего направления. В звезде узел~0 является центральным, поэтому все назначения доступны за один хоп.

\subsection{Примеры маршрутов}

\begin{table}[H]
\centering
\caption{Маршруты от узла 0 к узлам 5 и 9}
\label{tab:routes}
\begin{tabular}{lll}
\toprule
Топология & $0 \to 5$ & $0 \to 9$ \\
\midrule
Линейная & $0 \to 1 \to 2 \to 3 \to 4 \to 5$ & $0 \to 1 \to 2 \to \ldots \to 9$ \\
Кольцо   & $0 \to 1 \to 2 \to 3 \to 4 \to 5$ & $0 \to 9$ \\
Звезда   & $0 \to 5$                          & $0 \to 9$ \\
\bottomrule
\end{tabular}
\end{table}

Маршрут $0 \to 9$ в кольце проходит в один хоп, что подтверждает корректность выбора кратчайшего направления алгоритмом Дейкстры.

\subsection{Стохастические разрывы связей}

В данном эксперименте каждый канал сети независимо выходит из строя с вероятностью $p$. После отказов маршрутизаторы выполняют повторную сходимость, после чего измеряется доля достижимых пар узлов. Результаты усреднены по 20 запускам.

\begin{table}[H]
\centering
\caption{Стохастические разрывы ($N = 10$)}
\label{tab:failures}
\small
\begin{tabular}{l|ccc|ccc|ccc}
\toprule
& \multicolumn{3}{c|}{Линейная (9 кан.)} & \multicolumn{3}{c|}{Кольцо (10 кан.)} & \multicolumn{3}{c}{Звезда (9 кан.)} \\
$p$ & Разр. & Связн. & Такты & Разр. & Связн. & Такты & Разр. & Связн. & Такты \\
\midrule
0.00 & 0.0 & 1.00 & 10.0 & 0.0 & 1.00 & 6.0 & 0.0 & 1.00 & 3.0 \\
0.10 & 0.9 & 0.73 & 8.3  & 0.9 & 0.92 & 8.2 & 0.9 & 0.84 & 3.0 \\
0.20 & 1.6 & 0.59 & 7.2  & 1.7 & 0.72 & 6.9 & 1.6 & 0.71 & 3.0 \\
0.30 & 2.5 & 0.43 & 5.8  & 2.8 & 0.52 & 5.8 & 2.5 & 0.58 & 3.0 \\
0.50 & 4.3 & 0.23 & 3.9  & 4.8 & 0.27 & 4.1 & 4.3 & 0.35 & 3.0 \\
\bottomrule
\end{tabular}
\end{table}

{\small \textbf{Разр.} --- среднее число разорванных каналов; \textbf{Связн.} --- доля достижимых пар; \textbf{Такты} --- время пересходимости.}

Во всех 300 запусках (3 топологии $\times$ 5 вероятностей $\times$ 20 повторов) таблицы маршрутизации после повторной сходимости прошли верификацию --- расхождений с эталонными BFS-маршрутами не обнаружено.

\begin{figure}[H]
\centering
\begin{tikzpicture}
\begin{axis}[
    width=12cm, height=7cm,
    xlabel={Вероятность разрыва $p$},
    ylabel={Доля достижимых пар},
    legend pos=south west,
    grid=major,
    ymin=0, ymax=1.05,
    xtick={0,0.1,0.2,0.3,0.5}
]
\addplot[blue, mark=square*, thick] coordinates {
    (0,1.00) (0.10,0.73) (0.20,0.59) (0.30,0.43) (0.50,0.23)
};
\addlegendentry{Линейная}

\addplot[red, mark=triangle*, thick] coordinates {
    (0,1.00) (0.10,0.92) (0.20,0.72) (0.30,0.52) (0.50,0.27)
};
\addlegendentry{Кольцо}

\addplot[green!60!black, mark=diamond*, thick] coordinates {
    (0,1.00) (0.10,0.84) (0.20,0.71) (0.30,0.58) (0.50,0.35)
};
\addlegendentry{Звезда}

\end{axis}
\end{tikzpicture}
\caption{Связность сети при стохастических разрывах}
\label{fig:reachability}
\end{figure}

При малых вероятностях отказа наибольшую устойчивость демонстрирует кольцо (92\% при $p=0{,}1$), поскольку между любой парой узлов существуют два независимых пути. Линейная топология наиболее уязвима: единственный разрыв канала разделяет сеть на два сегмента, что даёт 73\% связности уже при $p=0{,}1$.

При $p=0{,}5$ звезда превосходит кольцо по связности (35\% против 27\%). Это объясняется тем, что в звезде связность каждого периферийного узла зависит только от одного канала до центра, тогда как в кольце множественные разрывы приводят к фрагментации сети на изолированные сегменты.

\subsection{Разрыв и восстановление одного канала}

В данном эксперименте моделируется разрыв и последующее восстановление канала 0--1.

\begin{table}[H]
\centering
\caption{Разрыв и восстановление канала 0--1}
\label{tab:single_failure}
\begin{tabular}{llccc}
\toprule
Топология & Событие & Тактов & Достижимых пар & Корректность \\
\midrule
\multirow{3}{*}{Линейная} & Начальное состояние & 10 & 90 & Да \\
 & Разрыв & 9 & 72 & Да \\
 & Восстановление & 10 & 90 & Да \\
\midrule
\multirow{3}{*}{Кольцо} & Начальное состояние & 6 & 90 & Да \\
 & Разрыв & 10 & 90 & Да \\
 & Восстановление & 6 & 90 & Да \\
\midrule
\multirow{3}{*}{Звезда} & Начальное состояние & 3 & 90 & Да \\
 & Разрыв & 3 & 72 & Да \\
 & Восстановление & 3 & 90 & Да \\
\bottomrule
\end{tabular}
\end{table}

В кольце после разрыва канала 0--1 все 90 пар остаются достижимыми --- трафик перенаправляется по альтернативному пути. При этом топология фактически вырождается в линейную, и время сходимости возрастает с 6 до 10 тактов. В линейной топологии и в звезде разрыв канала 0--1 изолирует узел~1 от части сети, в результате чего 18 пар узлов становятся недостижимыми.

% =================================================================
\section{Анализ}

Результаты экспериментов позволяют сопоставить характеристики трёх топологий.

\textbf{Звезда} обеспечивает наименьшее время сходимости (3 такта при любом $N$) и кратчайшие маршруты, однако полностью зависит от центрального узла: его отказ приводит к полной потере связности.

\textbf{Кольцо} обладает избыточностью: одиночный разрыв канала не нарушает связность, поскольку между любой парой узлов существует альтернативный путь. Время сходимости составляет $\approx N/2$ тактов. Вместе с тем при множественных отказах устойчивость кольца снижается.

\textbf{Линейная топология} наиболее уязвима к отказам: любой единственный разрыв канала разделяет сеть на два несвязных сегмента.

По устойчивости к массовым сбоям при $p=0{,}1$ кольцо сохраняет 92\% связности, звезда --- 84\%, линейная --- 73\%. При $p=0{,}3$ все три топологии опускаются ниже 60\%, что свидетельствует о необходимости избыточных каналов связи в реальных сетях.

% =================================================================
\section{Выводы}

\begin{enumerate}
    \item Реализован симулятор протокола OSPF на C++ с механизмом рассылки LSA, алгоритмом Дейкстры и пошаговой маршрутизацией. Корректность таблиц маршрутизации подтверждена сравнением с эталонными BFS-маршрутами во всех экспериментах.

    \item Время сходимости определяется диаметром сети: $N$ тактов для линейной топологии, $\approx N/2$ для кольца, 3 такта для звезды независимо от числа узлов.

    \item Кольцо --- единственная из рассмотренных топологий, сохраняющая полную связность при одиночном разрыве канала.

    \item Звезда обеспечивает минимальное время сходимости и кратчайшие маршруты, однако представляет собой единую точку отказа.

    \item При стохастических разрывах с $p \geq 0{,}3$ все три топологии теряют более 40\% связности, что указывает на необходимость использования избыточных топологий (ячеистых, полносвязных) в реальных сетях.
\end{enumerate}

% =================================================================
\section*{Приложение: исходный код}
\addcontentsline{toc}{section}{Приложение: исходный код}

Исходный код симулятора доступен в репозитории:

\url{https://github.com/AleksandrShvartz/NetworksLabs}

\end{document}
