\documentclass[12pt,a4paper]{article}
\usepackage[utf8]{inputenc}
\usepackage[T2A]{fontenc}
\usepackage[russian]{babel}
\usepackage{amsmath,amssymb}
\usepackage{graphicx}
\usepackage{booktabs}
\usepackage{multirow}
\usepackage{geometry}
\usepackage{float}
\usepackage{caption}
\usepackage{hyperref}
\usepackage{pgfplots}
\pgfplotsset{compat=1.18}
\usepackage{listings}
\usepackage{xcolor}

\geometry{left=2cm,right=2cm,top=2cm,bottom=2cm}

\lstset{
    language=C++,
    basicstyle=\ttfamily\small,
    keywordstyle=\color{blue},
    commentstyle=\color{green!50!black},
    stringstyle=\color{red},
    numbers=left,
    numberstyle=\tiny,
    breaklines=true,
    frame=single
}

\begin{document}

\begin{titlepage}
    \begin{center}
        \textsc{
            Санкт-Петербургский политехнический университет имени Петра Великого \\[5mm]
            Институт прикладной математики и механики\\[2mm]
            Высшая школа прикладной математики и физики
        }
        \vfill
        \textbf{\large
            Отчет \\
            по лабораторной работе \#1 \\
            по дисциплине \\
            ``Компьютерные сети''\\[5mm]
        }
        \textbf{\Large
            Реализация протоколов автоматического запроса\\
            повторной передачи Go-Back-N и Selective Repeat\\
        }
    \end{center}

    \vfill
    \hfill
    \begin{minipage}{0.5\textwidth}
        Выполнил: \vspace{1ex} \\
		Студент: Шварц Александр \\
		Группа: 5040102/40201\\
    \end{minipage}

	\hfill
	\begin{minipage}{0.5\textwidth}
		Принял: \vspace{1ex} \\
		к. ф.-м. н., доцент \\
		Баженов Александр Николаевич
	\end{minipage}

    \vfill
    \begin{center}
        Санкт-Петербург \\
        \the\year{} г.
    \end{center}
\end{titlepage}

\tableofcontents
\newpage

% =================================================================
\section{Введение}

Протоколы автоматического запроса повторной передачи (ARQ) обеспечивают надёжную доставку данных по ненадёжным каналам связи. В данной работе реализованы и сравнены два протокола скользящего окна:

\begin{itemize}
    \item \textbf{Go-Back-N (GBN)} --- при потере пакета отправитель повторно передаёт все пакеты, начиная с потерянного;
    \item \textbf{Selective Repeat (SR)} --- при потере пакета повторно передаётся только потерянный пакет.
\end{itemize}

Цель работы --- реализовать оба протокола, провести эксперименты с различными параметрами (вероятность потери, размер окна) и сравнить их эффективность.

% =================================================================
\section{Теоретические основы}

\subsection{Go-Back-N}

В протоколе GBN отправитель может иметь до $N$ неподтверждённых пакетов. Получатель принимает пакеты строго по порядку: если пакет с ожидаемым номером получен, отправляется кумулятивное подтверждение (ACK); если пришёл пакет не по порядку, он отбрасывается.

При тайм-ауте отправитель повторно передаёт \emph{все} пакеты в окне, начиная с базового. Это приводит к значительному количеству повторных передач при высоких вероятностях ошибок.

\subsection{Selective Repeat}

В протоколе SR получатель буферизует пакеты, пришедшие не по порядку, и отправляет индивидуальные ACK для каждого полученного пакета. При тайм-ауте отправитель повторно передаёт только конкретный неподтверждённый пакет.

Это значительно снижает количество повторных передач по сравнению с GBN, но требует буфера на стороне получателя.

% =================================================================
\section{Реализация}

Система реализована на языке C++ и включает следующие компоненты:

\begin{itemize}
    \item \texttt{Packet} --- структура пакета (номер последовательности, данные, флаг ACK);
    \item \texttt{Channel} --- канал связи с заданной вероятностью потери;
    \item \texttt{GBN\_Sender}, \texttt{GBN\_Receiver} --- отправитель и получатель GBN;
    \item \texttt{SR\_Sender}, \texttt{SR\_Receiver} --- отправитель и получатель SR.
\end{itemize}

Моделирование выполняется пошагово (tick-based): на каждом шаге отправитель посылает пакеты, получатель обрабатывает их и отправляет подтверждения, отправитель принимает ACK и проверяет тайм-ауты.

\textbf{Допущения модели.} Симулятор использует абстрактную модель с мгновенной доставкой (RTT~$= 0$): отправка пакета, его приём получателем и возврат ACK происходят в рамках одного такта. В такой модели размер окна $W$ выступает как множитель мгновенной пропускной способности (количество пакетов, отправляемых за один такт), а не как классическое сетевое окно, компенсирующее задержку канала. Это упрощение не влияет на сравнение эффективности протоколов (доля полезных передач $\eta$), однако абсолютные значения пропускной способности (пакетов/такт) следует интерпретировать с учётом данного допущения.

\subsection{Параметры эксперимента}

\begin{itemize}
    \item Количество передаваемых пакетов: $1000$
    \item Тайм-аут: $10$ тактов
    \item Размеры окна: $\{1, 2, 4, 8, 16, 32\}$
    \item Вероятности потери: $\{0; 0{,}05; 0{,}10; 0{,}20; 0{,}30; 0{,}50\}$
    \item Количество запусков для усреднения: $5$
\end{itemize}

% =================================================================
\section{Результаты экспериментов}

\subsection{Эффективность (доля полезных передач)}

Эффективность определяется как отношение числа уникальных пакетов к общему числу отправленных пакетов:
$$\eta = \frac{N_{\text{уник}}}{N_{\text{всего}}}$$

\begin{table}[H]
\centering
\caption{Эффективность протоколов при различных параметрах}
\label{tab:efficiency}
\small
\begin{tabular}{c|cc|cc|cc|cc}
\toprule
& \multicolumn{2}{c|}{$p=0{,}05$} & \multicolumn{2}{c|}{$p=0{,}10$} & \multicolumn{2}{c|}{$p=0{,}20$} & \multicolumn{2}{c}{$p=0{,}50$} \\
$W$ & GBN & SR & GBN & SR & GBN & SR & GBN & SR \\
\midrule
1  & 0.901 & 0.901 & 0.805 & 0.805 & 0.639 & 0.639 & 0.255 & 0.255 \\
2  & 0.899 & 0.901 & 0.796 & 0.806 & 0.635 & 0.639 & 0.279 & 0.255 \\
4  & 0.823 & 0.901 & 0.698 & 0.806 & 0.499 & 0.639 & 0.197 & 0.255 \\
8  & 0.700 & 0.901 & 0.529 & 0.805 & 0.342 & 0.640 & 0.111 & 0.256 \\
16 & 0.541 & 0.901 & 0.368 & 0.806 & 0.202 & 0.639 & 0.060 & 0.255 \\
32 & 0.360 & 0.901 & 0.227 & 0.806 & 0.109 & 0.640 & 0.032 & 0.258 \\
\bottomrule
\end{tabular}
\end{table}

\subsection{Пропускная способность (пакетов за такт)}

\begin{table}[H]
\centering
\caption{Пропускная способность протоколов}
\label{tab:throughput}
\small
\begin{tabular}{c|cc|cc|cc|cc}
\toprule
& \multicolumn{2}{c|}{$p=0{,}05$} & \multicolumn{2}{c|}{$p=0{,}10$} & \multicolumn{2}{c|}{$p=0{,}20$} & \multicolumn{2}{c}{$p=0{,}50$} \\
$W$ & GBN & SR & GBN & SR & GBN & SR & GBN & SR \\
\midrule
1  & 0.476 & 0.476 & 0.293 & 0.293 & 0.151 & 0.151 & 0.033 & 0.033 \\
2  & 0.949 & 0.673 & 0.565 & 0.395 & 0.300 & 0.206 & 0.075 & 0.047 \\
4  & 1.290 & 0.909 & 0.770 & 0.545 & 0.369 & 0.288 & 0.097 & 0.072 \\
8  & 1.563 & 1.282 & 0.831 & 0.774 & 0.404 & 0.442 & 0.099 & 0.114 \\
16 & 1.734 & 1.979 & 0.907 & 1.174 & 0.401 & 0.683 & 0.101 & 0.184 \\
32 & 1.742 & 2.957 & 0.917 & 1.936 & 0.386 & 1.122 & 0.105 & 0.310 \\
\bottomrule
\end{tabular}
\end{table}

\subsection{Количество повторных передач}

\begin{table}[H]
\centering
\caption{Среднее число повторных передач}
\label{tab:retransmissions}
\small
\begin{tabular}{c|cc|cc|cc|cc}
\toprule
& \multicolumn{2}{c|}{$p=0{,}05$} & \multicolumn{2}{c|}{$p=0{,}10$} & \multicolumn{2}{c|}{$p=0{,}20$} & \multicolumn{2}{c}{$p=0{,}50$} \\
$W$ & GBN & SR & GBN & SR & GBN & SR & GBN & SR \\
\midrule
1   & 110  & 110  & 242  & 242  & 565  & 565  & 2930  & 2930 \\
2   & 113  & 110  & 257  & 241  & 575  & 565  & 2583  & 2930 \\
4   & 216  & 110  & 434  & 242  & 1005 & 565  & 4078  & 2921 \\
8   & 430  & 110  & 895  & 242  & 1925 & 564  & 8051  & 2907 \\
16  & 854  & 110  & 1735 & 241  & 3976 & 566  & 15779 & 2922 \\
32  & 1788 & 110  & 3421 & 242  & 8210 & 564  & 30202 & 2883 \\
\bottomrule
\end{tabular}
\end{table}

% =================================================================
\section{Графики}

\subsection{Зависимость эффективности от вероятности потери}

\begin{figure}[H]
\centering
\begin{tikzpicture}
\begin{axis}[
    width=14cm, height=8cm,
    xlabel={Вероятность потери $p$},
    ylabel={Эффективность $\eta$},
    legend pos=north east,
    grid=major,
    ymin=0, ymax=1.05,
    xtick={0,0.05,0.1,0.2,0.3,0.5}
]
% GBN W=4
\addplot[blue, mark=square*, thick] coordinates {
    (0,1.0) (0.05,0.823) (0.10,0.698) (0.20,0.499) (0.30,0.365) (0.50,0.197)
};
\addlegendentry{GBN, $W=4$}

% GBN W=16
\addplot[blue, mark=triangle*, dashed, thick] coordinates {
    (0,1.0) (0.05,0.541) (0.10,0.368) (0.20,0.202) (0.30,0.128) (0.50,0.060)
};
\addlegendentry{GBN, $W=16$}

% GBN W=32
\addplot[blue, mark=diamond*, dotted, thick] coordinates {
    (0,1.0) (0.05,0.360) (0.10,0.227) (0.20,0.109) (0.30,0.066) (0.50,0.032)
};
\addlegendentry{GBN, $W=32$}

% SR W=4
\addplot[red, mark=square*, thick] coordinates {
    (0,1.0) (0.05,0.901) (0.10,0.806) (0.20,0.639) (0.30,0.491) (0.50,0.255)
};
\addlegendentry{SR, $W=4$}

% SR W=16
\addplot[red, mark=triangle*, dashed, thick] coordinates {
    (0,1.0) (0.05,0.901) (0.10,0.806) (0.20,0.639) (0.30,0.490) (0.50,0.255)
};
\addlegendentry{SR, $W=16$}

% SR W=32
\addplot[red, mark=diamond*, dotted, thick] coordinates {
    (0,1.0) (0.05,0.901) (0.10,0.806) (0.20,0.640) (0.30,0.493) (0.50,0.258)
};
\addlegendentry{SR, $W=32$}

\end{axis}
\end{tikzpicture}
\caption{Эффективность протоколов GBN и SR в зависимости от вероятности потери}
\label{fig:eff_vs_loss}
\end{figure}

\subsection{Зависимость пропускной способности от размера окна}

\begin{figure}[H]
\centering
\begin{tikzpicture}
\begin{axis}[
    width=14cm, height=8cm,
    xlabel={Размер окна $W$},
    ylabel={Пропускная способность (пакетов/такт)},
    legend pos=north west,
    grid=major,
    ymin=0,
    xmode=log,
    log basis x=2,
    xtick={1,2,4,8,16,32},
    xticklabels={1,2,4,8,16,32}
]
% p=0.10
\addplot[blue, mark=square*, thick] coordinates {
    (1,0.293) (2,0.565) (4,0.770) (8,0.831) (16,0.907) (32,0.917)
};
\addlegendentry{GBN, $p=0{,}10$}

\addplot[red, mark=square*, thick] coordinates {
    (1,0.293) (2,0.395) (4,0.545) (8,0.774) (16,1.174) (32,1.936)
};
\addlegendentry{SR, $p=0{,}10$}

% p=0.30
\addplot[blue, mark=triangle*, dashed, thick] coordinates {
    (1,0.088) (2,0.187) (4,0.222) (8,0.227) (16,0.234) (32,0.223)
};
\addlegendentry{GBN, $p=0{,}30$}

\addplot[red, mark=triangle*, dashed, thick] coordinates {
    (1,0.088) (2,0.123) (4,0.182) (8,0.279) (16,0.438) (32,0.712)
};
\addlegendentry{SR, $p=0{,}30$}

\end{axis}
\end{tikzpicture}
\caption{Пропускная способность в зависимости от размера окна}
\label{fig:tp_vs_window}
\end{figure}

\subsection{Количество повторных передач}

\begin{figure}[H]
\centering
\begin{tikzpicture}
\begin{axis}[
    width=14cm, height=8cm,
    xlabel={Размер окна $W$},
    ylabel={Повторные передачи},
    legend pos=north west,
    grid=major,
    ymin=0,
    xtick={1,2,4,8,16,32},
    ymode=log,
    log basis y=10
]
% p=0.10
\addplot[blue, mark=square*, thick] coordinates {
    (1,242) (2,257) (4,434) (8,895) (16,1735) (32,3421)
};
\addlegendentry{GBN, $p=0{,}10$}

\addplot[red, mark=square*, thick] coordinates {
    (1,242) (2,241) (4,242) (8,242) (16,241) (32,242)
};
\addlegendentry{SR, $p=0{,}10$}

% p=0.30
\addplot[blue, mark=triangle*, dashed, thick] coordinates {
    (1,1039) (2,987) (4,1741) (8,3487) (16,6825) (32,14204)
};
\addlegendentry{GBN, $p=0{,}30$}

\addplot[red, mark=triangle*, dashed, thick] coordinates {
    (1,1039) (2,1037) (4,1038) (8,1027) (16,1042) (32,1031)
};
\addlegendentry{SR, $p=0{,}30$}

\end{axis}
\end{tikzpicture}
\caption{Количество повторных передач (логарифмическая шкала)}
\label{fig:retrans}
\end{figure}

% =================================================================
\section{Анализ результатов}

\subsection{Влияние размера окна}

\textbf{GBN:} При увеличении размера окна эффективность GBN \emph{снижается} при наличии потерь. Это объясняется тем, что при тайм-ауте GBN повторно передаёт все $W$ пакетов в окне. Чем больше $W$, тем больше избыточных повторных передач. Например, при $p = 0{,}10$ и $W = 32$ эффективность составляет лишь $0{,}227$, тогда как при $W = 4$ --- $0{,}698$.

\textbf{SR:} Эффективность SR \emph{не зависит} от размера окна, так как повторно передаётся только потерянный пакет. При $p = 0{,}10$ эффективность стабильно составляет $\approx 0{,}806$ для любого $W$.

\subsection{Влияние вероятности потери}

При увеличении вероятности потери оба протокола теряют эффективность, однако GBN деградирует значительно быстрее, особенно при больших окнах:
\begin{itemize}
    \item При $p = 0{,}50$ и $W = 32$: GBN отправляет в среднем $31\,202$ пакетов для передачи $1000$ уникальных (эффективность $3{,}2\%$), в то время как SR~--- $3883$ пакета (эффективность $25{,}8\%$).
\end{itemize}

\subsection{Пропускная способность}

Для GBN пропускная способность достигает плато при увеличении $W$, так как выигрыш от параллельной передачи нивелируется ростом повторных передач. Для SR пропускная способность растёт с увеличением $W$, однако рост является \emph{сублинейным}: при $p = 0{,}10$ увеличение окна с $W = 1$ до $W = 32$ даёт рост пропускной способности лишь в $\approx 6{,}6$ раз (с $0{,}293$ до $1{,}936$), а не в $32$ раза. Это объясняется тем, что при больших окнах вероятность потерять хотя бы один пакет из окна стремится к единице, и отправитель регулярно простаивает в ожидании тайм-аута.

\subsection{Случай $W = 1$}

При $W = 1$ оба протокола эквивалентны протоколу Stop-and-Wait, что подтверждается одинаковыми результатами в экспериментах.

% =================================================================
\section{Выводы}

\begin{enumerate}
    \item \textbf{Selective Repeat превосходит Go-Back-N} по эффективности использования канала при любых условиях с потерями ($p > 0$).

    \item \textbf{Преимущество SR растёт} с увеличением размера окна и вероятности потери. При $W = 32$ и $p = 0{,}50$ SR в $8$ раз эффективнее GBN.

    \item \textbf{Эффективность SR не зависит от $W$}, тогда как эффективность GBN обратно пропорциональна размеру окна при наличии потерь.

    \item \textbf{Пропускная способность SR растёт сублинейно} с увеличением размера окна: при больших $W$ рост замедляется из-за простоев при тайм-аутах, однако SR остаётся предпочтительным для каналов с ненулевой вероятностью ошибок.

    \item \textbf{GBN проще в реализации}: не требует буфера на приёмной стороне и может быть предпочтительным при низких вероятностях потерь ($p < 0{,}05$) и малых окнах.

    \item При $W = 1$ оба протокола вырождаются в Stop-and-Wait и показывают одинаковую производительность.
\end{enumerate}

% =================================================================
\section*{Приложение: исходный код}
\addcontentsline{toc}{section}{Приложение: исходный код}

Исходный код симулятора доступен в репозитории:

\url{https://github.com/AleksandrShvartz/NetworksLabs}

\end{document}
